%%%%%%%%%%%%%%%%%%%%%%%%%%%%%%%%%%%%%%%%%%%%%%%%
%                                              %
%   ExempleLettre.tex                          %
%   Version 2.0                                %
%   1 janvier 2005                             %
%   Ce document est un exemple d'IRPHELetter   %
%                                              %
%   G.Searby     Juin 1999,   juillet 2001     %
%                 Mise � jour janvier 2005     %
%                                              %
%%%%%%%%%%%%%%%%%%%%%%%%%%%%%%%%%%%%%%%%%%%%%%%%
%
\documentclass[11pt,french]{IRPHELetter}
%\documentclass[11pt,english]{IRPHELetter}
% Taille optimale : 11pt.
%
\usepackage[latin1]{inputenc}     % Linux, Unix et OSX set for isolatin1
%\usepackage[ansinew]{inputenc}   % PC Windows
%\usepackage[applemac]{inputenc}  % Mac 0S9 et OSX set for MacOSroman
%
\name{Geoff Searby}                            % Nom exp�diteur 
\signature{G. Searby}                          % Signature exp�diteur
\myposition{IRPHE}                             % Appartenance
\mytitle{Directeur de Recherche au CNRS}       % Qualification de la signature
\tel{+33 496 13 97 77}                         % T�l�phone exp�diteur
\fax{+33 496 13 97 09}                         % Fax exp�diteur
\email{Geoff.Searby@irphe.univ-mrs.fr}         % M�l exp�dieur
\yref{Votre r�f�rence}                         % R�f�rence du destinataire
\ymail{Date de votre courrier}                 % Date courrier du destinataire
\myref{IRPHELetter v2.0 : janvier 2005 }       % Notre r�ference
\objet{Exemple d'utilisation d'IRPHELetter.cls}% Objet de la lettre
%\date{1er janvier 2005}                       % Par d�faut la date sera la date courante
                                               % D�commentez pour obtenir une date fixe
%
%
\begin{document}
%
% L'adresse du destinataire est mise entre {} apr�s "\begin{letter}"
%
\begin{letter}{
Coll�gues d'IRPHE\\
49, rue Fr�d�ric Joliot--Curie\\
Technopole de Ch�teau Gombert\\
13384 MARSEILLE Cedex 13\\
France}

\opening{Chers coll�gues,} 

Ce document est un exemple d'utilisation d'\texttt{IRPHELetter.cls}. 
Le document joint, \texttt{DocumentationIRPHELetter.tex} d�crit son utilisation en 
d�tail. Les options d'IRPHELettre permettent d'�crire en anglais et en 
fran�ais. La police de caract�res est en Sans S�rif. 
La taille optimale est 11pt. Cet exemple donne 
une illustration quasi-exhaustive des commandes annexes. Si ces 
commandes sont omises, la rubrique correspondant n'est pas imprim�e.


Lorem ipsum dolor sit amet, consectetuer adipiscing elit. Suspendisse nec felis eget dolor volutpat blandit. Sed pede. Cras iaculis nibh non augue. Integer egestas, nisl ultricies iaculis consectetuer, felis risus blandit nibh, nec dapibus eros ante placerat massa. Aliquam erat volutpat. Etiam quis eros sed quam ultrices tempor. Sed faucibus lacus in velit. Fusce dolor orci, ullamcorper ut, scelerisque eget, aliquam pretium, nibh. Suspendisse porttitor fringilla neque. Pellentesque bibendum, dolor in facilisis lobortis, orci diam congue elit, tempus adipiscing turpis eros a nibh. Pellentesque viverra viverra est. Duis lorem erat, sodales eu  magna. Proin in arcu sit amet quam adipiscing bibendum, sic ad nauseum.


%Duis autem vel eum iriure dolor in hendrerit in vulputate velit esse  molestie consequat, vel illum dolore eu feugiat nulla facilisis at vero  eros et accumsan et iusto odio dignissim qui blandit praesent luptatum  zzril delenit augue duis dolore te feugait nulla facilisi. Lorem ipsum dolor sit amet, consectetuer adipiscing elit, sed diam nonummy nibh  euismod tincidunt ut laoreet dolore magna aliquam erat volutpat.




\closing{Veuillez agr�er, mes chers coll�gues, l'assurance de mes sentiments 
distingu�s}
%\closing{Yours sincerely,}

\cc{Le directeur}
\encl{Documentation}
\ps{Ceci est un exemple de Post--Scriptum}

\end{letter}

\end{document}
