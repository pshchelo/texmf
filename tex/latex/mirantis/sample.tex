\documentclass[12pt]{article}
\usepackage[english]{babel}
\usepackage[
  left=1.5cm, right=1.5cm, top=2cm, bottom=2cm, headsep=0.5cm, footskip=0.6cm, bindingoffset=0cm
]{geometry}
\usepackage{fancyhdr}

\usepackage{tikz}
\usepackage{mirantis}

\begin{document}
\thispagestyle{empty}
\titleGWP{Aristotle}{Canonical Texts}{350 BC}



\section{METAPHYSICS}
ALL men by nature desire to know. An indication of this is the
delight we take in our senses; for even apart from their usefulness
they are loved for themselves; and above all others the sense of
sight. For not only with a view to action, but even when we are not
going to do anything, we prefer seeing (one might say) to everything
else. The reason is that this, most of all the senses, makes us know
and brings to light many differences between things.

    By nature animals are born with the faculty of sensation, and from
sensation memory is produced in some of them, though not in others.
And therefore the former are more intelligent and apt at learning than
those which cannot remember; those which are incapable of hearing
sounds are intelligent though they cannot be taught, e.\,g. the bee, and
any other race of animals that may be like it; and those which besides
memory have this sense of hearing can be taught.

    The animals other than man live by appearances and memories, and
have but little of connected experience; but the human race lives also
by art and reasonings. Now from memory experience is produced in
men; for the several memories of the same thing produce finally the
capacity for a single experience. And experience seems pretty much
like science and art, but really science and art come to men through
experience; for ``experience made art", as Polus says, ``but
inexperience luck". Now art arises when from many notions gained by
experience one universal judgement about a class of objects is
produced. For to have a judgement that when Callias was ill of this
disease this did him good, and similarly in the case of Socrates and
in many individual cases, is a matter of experience; but to judge that
it has done good to all persons of a certain constitution, marked
off in one class, when they were ill of this disease, e.\,g. to
phlegmatic or bilious people when burning with fevers-this is a matter
of art.

 ALL men by nature desire to know. An indication of this is the
delight we take in our senses; for even apart from their usefulness
they are loved for themselves; and above all others the sense of
sight. For not only with a view to action, but even when we are not
going to do anything, we prefer seeing (one might say) to everything
else. The reason is that this, most of all the senses, makes us know
and brings to light many differences between things.

    By nature animals are born with the faculty of sensation, and from
sensation memory is produced in some of them, though not in others.
And therefore the former are more intelligent and apt at learning than
those which cannot remember; those which are incapable of hearing
sounds are intelligent though they cannot be taught, e.\,g. the bee, and
any other race of animals that may be like it; and those which besides
memory have this sense of hearing can be taught.

    The animals other than man live by appearances and memories, and
have but little of connected experience; but the human race lives also
by art and reasonings. Now from memory experience is produced in
men; for the several memories of the same thing produce finally the
capacity for a single experience. And experience seems pretty much
like science and art, but really science and art come to men through
experience; for ``experience made art", as Polus says, ``but
inexperience luck". Now art arises when from many notions gained by
experience one universal judgement about a class of objects is
produced. For to have a judgement that when Callias was ill of this
disease this did him good, and similarly in the case of Socrates and
in many individual cases, is a matter of experience; but to judge that
it has done good to all persons of a certain constitution, marked
off in one class, when they were ill of this disease, e.\,g. to
phlegmatic or bilious people when burning with fevers-this is a matter
of art.

\section{NICOMACHEAN ETHICS}

\subsection{}
  EVERY art and every inquiry, and similarly every action and pursuit,
is thought to aim at some good; and for this reason the good has
rightly been declared to be that at which all things aim. But a
certain difference is found among ends; some are activities, others
are products apart from the activities that produce them. Where
there are ends apart from the actions, it is the nature of the
products to be better than the activities. Now, as there are many
actions, arts, and sciences, their ends also are many; the end of
the medical art is health, that of shipbuilding a vessel, that of
strategy victory, that of economics wealth. But where such arts fall
under a single capacity -- as bridle-making and the other arts concerned
with the equipment of horses fall under the art of riding, and this
and every military action under strategy, in the same way other arts
fall under yet others -- in all of these the ends of the master arts
are to be preferred to all the subordinate ends; for it is for the
sake of the former that the latter are pursued. It makes no difference
whether the activities themselves are the ends of the actions, or
something else apart from the activities, as in the case of the
sciences just mentioned.

\subsection{}
  If, then, there is some end of the things we do, which we desire for
its own sake (everything else being desired for the sake of this), and
if we do not choose everything for the sake of something else (for
at that rate the process would go on to infinity, so that our desire
would be empty and vain), clearly this must be the good and the
chief good. Will not the knowledge of it, then, have a great influence
on life? Shall we not, like archers who have a mark to aim at, be more
likely to hit upon what is right? If so, we must try, in outline at
least, to determine what it is, and of which of the sciences or
capacities it is the object. It would seem to belong to the most
authoritative art and that which is most truly the master art. And
politics appears to be of this nature; for it is this that ordains
which of the sciences should be studied in a state, and which each
class of citizens should learn and up to what point they should
learn them; and we see even the most highly esteemed of capacities
to fall under this, e.\,g. strategy, economics, rhetoric; now, since
politics uses the rest of the sciences, and since, again, it
legislates as to what we are to do and what we are to abstain from,
the end of this science must include those of the others, so that this
end must be the good for man. For even if the end is the same for a
single man and for a state, that of the state seems at all events
something greater and more complete whether to attain or to
preserve; though it is worth while to attain the end merely for one
man, it is finer and more godlike to attain it for a nation or for
city-states. These, then, are the ends at which our inquiry aims,
since it is political science, in one sense of that term.

\subsection{}
  Our discussion will be adequate if it has as much clearness as the
subject-matter admits of, for precision is not to be sought for
alike in all discussions, any more than in all the products of the
crafts. Now fine and just actions, which political science
investigates, admit of much variety and fluctuation of opinion, so
that they may be thought to exist only by convention, and not by
nature. And goods also give rise to a similar fluctuation because they
bring harm to many people; for before now men have been undone by
reason of their wealth, and others by reason of their courage. We must
be content, then, in speaking of such subjects and with such premisses
to indicate the truth roughly and in outline, and in speaking about
things which are only for the most part true and with premisses of the
same kind to reach conclusions that are no better. In the same spirit,
therefore, should each type of statement be received; for it is the
mark of an educated man to look for precision in each class of
things just so far as the nature of the subject admits; it is
evidently equally foolish to accept probable reasoning from a
mathematician and to demand from a rhetorician scientific proofs.

  Now each man judges well the things he knows, and of these he is a
good judge. And so the man who has been educated in a subject is a
good judge of that subject, and the man who has received an
all-round education is a good judge in general. Hence a young man is
not a proper hearer of lectures on political science; for he is
inexperienced in the actions that occur in life, but its discussions
start from these and are about these; and, further, since he tends
to follow his passions, his study will be vain and unprofitable,
because the end aimed at is not knowledge but action. And it makes
no difference whether he is young in years or youthful in character;
the defect does not depend on time, but on his living, and pursuing
each successive object, as passion directs. For to such persons, as to
the incontinent, knowledge brings no profit; but to those who desire
and act in accordance with a rational principle knowledge about such
matters will be of great benefit.

  These remarks about the student, the sort of treatment to be
expected, and the purpose of the inquiry, may be taken as our preface.

\subsection{}
  Let us resume our inquiry and state, in view of the fact that all
knowledge and every pursuit aims at some good, what it is that we
say political science aims at and what is the highest of all goods
achievable by action. Verbally there is very general agreement; for
both the general run of men and people of superior refinement say that
it is happiness, and identify living well and doing well with being
happy; but with regard to what happiness is they differ, and the
many do not give the same account as the wise. For the former think it
is some plain and obvious thing, like pleasure, wealth, or honour;
they differ, however, from one another -- and often even the same man
identifies it with different things, with health when he is ill,
with wealth when he is poor; but, conscious of their ignorance, they
admire those who proclaim some great ideal that is above their
comprehension. Now some thought that apart from these many goods there
is another which is self-subsistent and causes the goodness of all
these as well. To examine all the opinions that have been held were
perhaps somewhat fruitless; enough to examine those that are most
prevalent or that seem to be arguable.

  Let us not fail to notice, however, that there is a difference
between arguments from and those to the first principles. For Plato,
too, was right in raising this question and asking, as he used to
do, ``are we on the way from or to the first principles?" There is a
difference, as there is in a race-course between the course from the
judges to the turning-point and the way back. For, while we must begin
with what is known, things are objects of knowledge in two senses --
some to us, some without qualification. Presumably, then, we must
begin with things known to us. Hence any one who is to listen
intelligently to lectures about what is noble and just, and generally,
about the subjects of political science must have been brought up in
good habits. For the fact is the starting-point, and if this is
sufficiently plain to him, he will not at the start need the reason as
well; and the man who has been well brought up has or can easily get
startingpoints.


\section{ON INTERPRETATION}

\subsection{}

  First we must define the terms `noun' and `verb', then the terms
`denial' and `affirmation', then `proposition' and `sentence'.

  Spoken words are the symbols of mental experience and written
words are the symbols of spoken words. Just as all men have not the
same writing, so all men have not the same speech sounds, but the
mental experiences, which these directly symbolize, are the same for
all, as also are those things of which our experiences are the images.
This matter has, however, been discussed in my treatise about the
soul, for it belongs to an investigation distinct from that which lies
before us.

  As there are in the mind thoughts which do not involve truth or
falsity, and also those which must be either true or false, so it is
in speech. For truth and falsity imply combination and separation.
Nouns and verbs, provided nothing is added, are like thoughts
without combination or separation; `man' and `white', as isolated
terms, are not yet either true or false. In proof of this, consider
the word `goat-stag'. It has significance, but there is no truth or
falsity about it, unless `is' or `is not' is added, either in the
present or in some other tense.

\subsection{}

  By a noun we mean a sound significant by convention, which has no
reference to time, and of which no part is significant apart from
the rest. In the noun `Fairsteed', the part `steed' has no
significance in and by itself, as in the phrase `fair steed'. Yet
there is a difference between simple and composite nouns; for in the
former the part is in no way significant, in the latter it contributes
to the meaning of the whole, although it has not an independent
meaning. Thus in the word `pirate-boat' the word `boat' has no meaning
except as part of the whole word.

  The limitation `by convention' was introduced because nothing is
by nature a noun or name-it is only so when it becomes a symbol;
inarticulate sounds, such as those which brutes produce, are
significant, yet none of these constitutes a noun.

  The expression `not-man' is not a noun. There is indeed no
recognized term by which we may denote such an expression, for it is
not a sentence or a denial. Let it then be called an indefinite noun.

  The expressions `of Philo', `to Philo', and so on, constitute not
nouns, but cases of a noun. The definition of these cases of a noun is
in other respects the same as that of the noun proper, but, when
coupled with `is', `was', or `will be', they do not, as they are,
form a proposition either true or false, and this the noun proper
always does, under these conditions. Take the words `of Philo is' or
 `of Philo is not'; these words do not, as they stand, form
either a true or a false proposition.

\subsection{}

  A verb is that which, in addition to its proper meaning, carries
with it the notion of time. No part of it has any independent meaning,
and it is a sign of something said of something else.

  I will explain what I mean by saying that it carries with it the
notion of time. `Health' is a noun, but `is healthy' is a verb; for
besides its proper meaning it indicates the present existence of the
state in question.

  Moreover, a verb is always a sign of something said of something
else, i.\,e. of something either predicable of or present in some
other thing.

  Such expressions as `is not-healthy', `is not ill', I do not
describe as verbs; for though they carry the additional note of
time, and always form a predicate, there is no specified name for this
variety; but let them be called indefinite verbs, since they apply
equally well to that which exists and to that which does not.

  Similarly `he was healthy', `he will be healthy', are not verbs, but
tenses of a verb; the difference lies in the fact that the verb
indicates present time, while the tenses of the verb indicate those
times which lie outside the present.

  Verbs in and by themselves are substantival and have significance,
for he who uses such expressions arrests the hearer's mind, and
fixes his attention; but they do not, as they stand, express any
judgement, either positive or negative. For neither are `to be' and
`not to be' the participle `being' significant of any fact, unless
something is added; for they do not themselves indicate anything,
but imply a copulation, of which we cannot form a conception apart
from the things coupled.

\chapter{Francis Bacon}

\section{THE ESSAYS}

\subsection{OF TRUTH}

What is truth? said jesting Pilate, and would not stay for an
answer. Certainly there be, that delight in giddiness, and count it
a bondage to fix a belief; affecting free-will in thinking, as well as
in acting. And though the sects of philosophers of that kind be
gone, yet there remain certain discoursing wits, which are of the same
veins, though there be not so much blood in them, as was in those of
the ancients. But it is not only the difficulty and labor, which men
take in finding out of truth, nor again, that when it is found, it
imposeth upon men's thoughts, that doth bring lies in favor; but a
natural though corrupt love, of the lie itself. One of the later
school of the Grecians, examineth the matter, and is at a stand, to
think what should be in it, that men should love lies; where neither
they make for pleasure, as with poets, nor for advantage, as with
the merchant; but for the lie's sake. But I cannot tell; this same
truth, is a naked, and open day-light, that doth not show the masks,
and mummeries, and triumphs, of the world, half so stately and
daintily as candle-lights. Truth may perhaps come to the price of a
pearl, that showeth best by day; but it will not rise to the price
of a diamond, or carbuncle, that showeth best in varied lights. A
mixture of a lie doth ever add pleasure. Doth any man doubt, that if
there were taken out of men's minds, vain opinions, flattering
hopes, false valuations, imaginations as one would, and the like,
but it would leave the minds, of a number of men, poor shrunken
things, full of melancholy and indisposition, and unpleasing to
themselves?

One of the fathers, in great severity, called poesy vinum
doemonum, because it filleth the imagination; and yet, it is but
with the shadow of a lie. But it is not the lie that passeth through
the mind, but the lie that sinketh in, and settleth in it, that doth
the hurt; such as we spake of before. But, howsoever these things
are thus in men's depraved judgments, and affections, yet truth, which
only doth judge itself, teacheth that the inquiry of truth, which is
the love-making, or wooing of it, the knowledge of truth, which is the
presence of it, and the belief of truth, which is the enjoying of
it, is the sovereign good of human nature. The first creature of
God, in the works of the days, was the light of the sense; the last,
was the light of reason; and his sabbath work ever since, is the
illumination of his Spirit. First he breathed light, upon the face
of the matter or chaos; then he breathed light, into the face of
man; and still he breatheth and inspireth light, into the face of
his chosen. The poet, that beautified the sect, that was otherwise
inferior to the rest, saith yet excellently well: It is a pleasure, to
stand upon the shore, and to see ships tossed upon the sea; a
pleasure, to stand in the window of a castle, and to see a battle, and
the adventures thereof below: but no pleasure is comparable to the
standing upon the vantage ground of truth (a hill not to be commanded,
and where the air is always clear and serene), and to see the
errors, and wanderings, and mists, and tempests, in the vale below; so
always that this prospect be with pity, and not with swelling, or
pride. Certainly, it is heaven upon earth, to have a man's mind move
in charity, rest in providence, and turn upon the poles of truth.

To pass from theological, and philosophical truth, to the truth of
civil business; it will be acknowledged, even by those that practise
it not, that clear, and round dealing, is the honor of man's nature;
and that mixture of falsehoods, is like alloy in coin of gold and
silver, which may make the metal work the better, but it embaseth
it. For these winding, and crooked courses, are the goings of the
serpent; which goeth basely upon the belly, and not upon the feet.
There is no vice, that doth so cover a man with shame, as to be
found false and perfidious. And therefore Montaigne saith prettily,
when he inquired the reason, why the word of the lie should be such
a disgrace, and such an odious charge? Saith he, If it be well
weighed, to say that a man lieth, is as much to say, as that he is
brave towards God, and a coward towards men. For a lie faces God,
and shrinks from man. Surely the wickedness of falsehood, and breach
of faith, cannot possibly be so highly expressed, as in that it
shall be the last peal, to call the judgments of God upon the
generations of men; it being foretold, that when Christ cometh, he
shall not find faith upon the earth.

\subsection{OF DEATH}

Men fear death, as children fear to go in the dark; and as that
natural fear in children, is increased with tales, so is the other.
Certainly, the contemplation of death, as the wages of sin, and
passage to another world, is holy and religious; but the fear of it,
as a tribute due unto nature, is weak. Yet in religious meditations,
there is sometimes mixture of vanity, and of superstition. You shall
read, in some of the friars' books of mortification, that a man should
think with himself, what the pain is, if he have but his finger's
end pressed, or tortured, and thereby imagine, what the pains of death
are, when the whole body is corrupted, and dissolved; when many
times death passeth, with less pain than the torture of a limb; for
the most vital parts, are not the quickest of sense. And by him that
spake only as a philosopher, and natural man, it was well said,
Pompa mortis magis terret, quam mors ipsa. Groans, and convulsions,
and a discolored face, and friends weeping, and blacks, and obsequies,
and the like, show death terrible. It is worthy the observing, that
there is no passion in the mind of man, so weak, but it mates, and
masters, the fear of death; and therefore, death is no such terrible
enemy, when a man hath so many attendants about him, that can win
the combat of him. Revenge triumphs over death; love slights it; honor
aspireth to it; grief flieth to it; fear preoccupateth it; nay, we
read, after Otho the emperor had slain himself, pity (which is the
tenderest of affections) provoked many to die, out of mere
compassion to their sovereign, and as the truest sort of followers.
Nay, Seneca adds niceness and satiety: Cogita quamdiu eadem feceris;
mori velle, non tantum fortis aut miser, sed etiam fastidiosus potest.
A man would die, though he were neither valiant, nor miserable, only
upon a weariness to do the same thing so oft, over and over. It is
no less worthy, to observe, how little alteration in good spirits, the
approaches of death make; for they appear to be the same men, till the
last instant. Augustus Caesar died in a compliment; Livia, conjugii
nostri memor, vive et vale. Tiberius in dissimulation; as Tacitus
saith of him, Jam Tiberium vires et corpus, non dissimulatio,
deserebant. Vespasian in a jest, sitting upon the stool; Ut puto
deus fio. Galba with a sentence; Feri, si ex re sit populi Romani;
holding forth his neck. Septimius Severus in despatch; Adeste si
quid mihi restat agendum. And the like. Certainly the Stoics
bestowed too much cost upon death, and by their great preparations,
made it appear more fearful. Better saith he qui finem vitae
extremum inter munera ponat naturae. It is as natural to die, as to be
born; and to a little infant, perhaps, the one is as painful, as the
other. He that dies in an earnest pursuit, is like one that is wounded
in hot blood; who, for the time, scarce feels the hurt; and
therefore a mind fixed, and bent upon somewhat that is good, doth
avert the dolors of death. But, above all, believe it, the sweetest
canticle is, Nunc dimittis; when a man hath obtained worthy ends,
and expectations. Death hath this also; that it openeth the gate to
good fame, and extinguisheth envy. -- Extinctus amabitur idem.

\chapter{Immanuel Kant}
\section{THE CRITIQUE OF JUDGEMENT}
\subsection{Division of Philosophy}

  Philosophy may be said to contain the principles of the rational
cognition that concepts afford us of things (not merely, as with
logic, the principles of the form of thought in general irrespective
of the objects), and, thus interpreted, the course, usually adopted,
of dividing it into theoretical and practical is perfectly sound.
But this makes imperative a specific distinction on the part of the
concepts by which the principles of this rational cognition get
their object assigned to them, for if the concepts are not distinct
they fail to justify a division, which always presupposes that the
principles belonging to the rational cognition of the several parts of
the science in question are themselves mutually exclusive.

  Now there are but two kinds of concepts, and these yield a
corresponding number of distinct principles of the possibility of
their objects. The concepts referred to are those of nature and that
of freedom. By the first of these, a theoretical cognition from a
priori principles becomes possible. In respect of such cognition,
however, the second, by its very concept, imports no more than a
negative principle (that of simple antithesis), while for the
determination of the will, on the other hand, it establishes
fundamental principles which enlarge the scope of its activity, and
which on that account are called practical. Hence the division of
philosophy falls properly into two parts, quite distinct in their
principles -- a theoretical, as philosophy of nature, and a practical, as
philosophy of morals (for this is what the practical legislation of
reason by the concept of freedom is called). Hitherto, however, in the
application of these expressions to the division of the different
principles, and with them to the division of philosophy, a gross
misuse of the terms has prevailed; for what is practical according
to concepts of nature bas been taken as identical with what is
practical according to the concept of freedom, with the result that
a division has been made under these heads of theoretical and
practical, by which, in effect, there has been no division at all
(seeing that both parts might have similar principles).

  The will -- for this is what is said -- is the faculty of desire and, as
such, is just one of the many natural causes in the world, the one,
namely, which acts by concepts; and whatever is represented as
possible (or necessary) through the efficacy of will is called
practically possible (or necessary): the intention being to
distinguish its possibility (or necessity) from the physical
possibility or necessity of an effect the causality of whose cause
is not determined to its production by concepts (but rather, as with
lifeless matter, by mechanism, and, as with the lower animals, by
instinct). Now, the question in respect of the practical faculty:
whether, that is to say, the concept, by which the causality of the
will gets its rule, is a concept of nature or of freedom, is here left
quite open.

\subsection{The Realm of Philosophy in General}
The employment of our faculty of cognition from principles, and with
it philosophy, is coextensive with the applicability of a priori
concepts.

  Now a division of the complex of all the objects to which those
concepts are referred for the purpose, where possible, of compassing
their knowledge, may be made according to the varied competence or
incompetence of our faculty in that connection.

  Concepts, so far as they are referred to objects apart from the
question of whether knowledge of them is possible or not, have their
field, which is determined simply by the relation in which their
object stands to our faculty of cognition in general. The part of this
field in which knowledge is possible for us is a territory
(territorium) for these concepts and the requisite cognitive
faculty. The part of the territory over which they exercise
legislative authority is the realm (ditio) of these concepts, and
their appropriate cognitive faculty. Empirical concepts have,
therefore, their territory, doubtless, in nature as the complex of all
sensible objects, but they have no realm (only a dwelling-place,
domicilium), for, although they are formed according to law, they
are not themselves legislative, but the rules founded on them are
empirical and, consequently, contingent.

  Our entire faculty of cognition has two realms, that of natural
concepts and that of the concept of freedom, for through both it
prescribes laws a priori. In accordance with this distinction, then,
philosophy is divisible into theoretical and practical. But the
territory upon which its realm is established, and over which it
exercises its legislative authority, is still always confined to the
complex of the objects of all possible experience, taken as no more
than mere phenomena, for otherwise legislation by the understanding in
respect of them is unthinkable.

\section{THE CRITIQUE OF PRACTICAL REASON}
\subsection*{PREFACE}.

  This work is called the Critique of Practical Reason, not of the
pure practical reason, although its parallelism with the speculative
critique would seem to require the latter term. The reason of this
appears sufficiently from the treatise itself. Its business is to show
that there is pure practical reason, and for this purpose it
criticizes the entire practical faculty of reason. If it succeeds in
this, it has no need to criticize the pure faculty itself in order
to see whether reason in making such a claim does not presumptuously
overstep itself (as is the case with the speculative reason). For
if, as pure reason, it is actually practical, it proves its own
reality and that of its concepts by fact, and all disputation
against the possibility of its being real is futile.

  With this faculty, transcendental freedom is also established;
freedom, namely, in that absolute sense in which speculative reason
required it in its use of the concept of causality in order to
escape the antinomy into which it inevitably falls, when in the
chain of cause and effect it tries to think the unconditioned.
Speculative reason could only exhibit this concept (of freedom)
problematically as not impossible to thought, without assuring it
any objective reality, and merely lest the supposed impossibility of
what it must at least allow to be thinkable should endanger its very
being and plunge it into an abyss of scepticism.

  Inasmuch as the reality of the concept of freedom is proved by an
apodeictic law of practical reason, it is the keystone of the whole
system of pure reason, even the speculative, and all other concepts
(those of God and immortality) which, as being mere ideas, remain in
it unsupported, now attach themselves to this concept, and by it
obtain consistence and objective reality; that is to say, their
possibility is proved by the fact that freedom actually exists, for
this idea is revealed by the moral law.

\subsection{Of the Idea of a Critique of Practical Reason}

The theoretical use of reason was concerned with objects of the
cognitive faculty only, and a critical examination of it with
reference to this use applied properly only to the pure faculty of
cognition; because this raised the suspicion, which was afterwards
confirmed, that it might easily pass beyond its limits, and be lost
among unattainable objects, or even contradictory notions. It is quite
different with the practical use of reason. In this, reason is
concerned with the grounds of determination of the will, which is a
faculty either to produce objects corresponding to ideas, or to
determine ourselves to the effecting of such objects (whether the
physical power is sufficient or not); that is, to determine our
causality. For here, reason can at least attain so far as to determine
the will, and has always objective reality in so far as it is the
volition only that is in question. The first question here then is
whether pure reason of itself alone suffices to determine the will, or
whether it can be a ground of determination only as dependent on
empirical conditions. Now, here there comes in a notion of causality
justified by the critique of the pure reason, although not capable
of being presented empirically, viz., that of freedom; and if we can
now discover means of proving that this property does in fact belong
to the human will (and so to the will of all rational beings), then it
will not only be shown that pure reason can be practical, but that
it alone, and not reason empirically limited, is indubitably
practical; consequently, we shall have to make a critical examination,
not of pure practical reason, but only of practical reason
generally. For when once pure reason is shown to exist, it needs no
critical examination. For reason itself contains the standard for
the critical examination of every use of it. The critique, then, of
practical reason generally is bound to prevent the empirically
conditioned reason from claiming exclusively to furnish the ground
of determination of the will. If it is proved that there is a
[practical] reason, its employment is alone immanent; the
empirically conditioned use, which claims supremacy, is on the
contrary transcendent and expresses itself in demands and precepts
which go quite beyond its sphere. This is just the opposite of what
might be said of pure reason in its speculative employment.

  However, as it is still pure reason, the knowledge of which is
here the foundation of its practical employment, the general outline
of the classification of a critique of practical reason must be
arranged in accordance with that of the speculative. We must, then,
have the Elements and the Methodology of it; and in the former an
Analytic as the rule of truth, and a Dialectic as the exposition and
dissolution of the illusion in the judgements of practical reason. But
the order in the subdivision of the Analytic will be the reverse of
that in the critique of the pure speculative reason. For, in the
present case, we shall commence with the principles and proceed to the
concepts, and only then, if possible, to the senses; whereas in the
case of the speculative reason we began with the senses and had to end
with the principles. The reason of this lies again in this: that now
we have to do with a will, and have to consider reason, not in its
relation to objects, but to this will and its causality. We must,
then, begin with the principles of a causality not empirically
conditioned, after which the attempt can be made to establish our
notions of the determining grounds of such a will, of their
application to objects, and finally to the subject and its sense
faculty. We necessarily begin with the law of causality from
freedom, that is, with a pure practical principle, and this determines
the objects to which alone it can be applied.

\end{document} 